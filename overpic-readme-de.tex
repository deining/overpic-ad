\documentclass[german, pagesize=auto, fontsize=12pt, parskip=half, DIV=11]{scrartcl}

\usepackage{fixltx2e}
\usepackage{etex}
\usepackage{xspace}
\usepackage{lmodern}
\usepackage[T1]{fontenc}
\usepackage{textcomp}
\usepackage{babel}
\usepackage[utf8]{inputenc}
\usepackage{microtype}
\usepackage{hyperref}

\newcommand*{\mail}[1]{\href{mailto:#1}{\texttt{#1}}}
\newcommand*{\pkg}[1]{\textsf{#1}}
\newcommand*{\env}[1]{\texttt{#1}}
\newsavebox{\tempbox}

\addtokomafont{title}{\rmfamily}

\title{Das Paket \pkg{overpic}\thanks{Dieses Handbuch bezieht sich auf \pkg{overpic}~v0.52 vom~4.~Juli~1999.}}
\author{Rolf Niepraschk\thanks{\mail{niepraschk@ptb.de}}}
\date{4.~Juli~1999}


\begin{document}

\maketitle

\noindent

\begin{quote}
  \footnotesize
  Copyright  1999 Rolf Niepraschk, \mail{niepraschk@ptb.de}\\
  This program can be redistributed and/or modified under the terms
  of the \LaTeX\ Project Public License Distributed from CTAN
  archives in directory \href{http://ctan.org/macros/latex/base/lppl.txt}{\texttt{macros/latex/base\slash lppl.txt}}; either
  version~1 of the License, or any later version.
\end{quote}

Dieses kleine \LaTeX-Paket definiert die \env{overpic}-Umgebung, welche eine
Kombination von \env{picture}-Umgebung und \env{includegraphics}-Befehl ist. Die
resultierende \env{picture}-Umgebung hat dieselbe Groesse wie die eingefuegte Grafik.
Jetzt ist es einfach moeglich beliebige \LaTeX-Ausgaben auf das Bild zu
positionieren. Ein Gitter kann zur Hilfe verwendet werden.

\begin{lrbox}{\tempbox}
  \begin{tabular}[b]{@{}l@{}}
    \texttt{opic-rel.tex} \\
    und                   \\
    \texttt{opic-abs.tex}
  \end{tabular}
\end{lrbox}

Das Verzeichnis enthaelt:
%
\begin{labeling}[\hspace{\labelsep}--]{\usebox{\tempbox}}
\item[\texttt{README.de}] Diese Datei.
\item[\texttt{README}] Das gleiche in englisch.
\item[\texttt{overpic.sty}] Die Definition von \env{overpic}. Benoetigt \pkg{graphicx.sty} und \pkg{epic.sty}.
\item[\usebox{\tempbox}] Zwei Testdateien, die alle Moeglichkeiten zeigen. Die Grafikdatei \texttt{golfer.ps} wird benoetigt. Sie ist in \textsc{ghostscript}s \texttt{examples}-Verzeichnis enthalten.
\end{labeling}
               
Kommentare und Vorschlaege bitte an \mail{niepraschk@ptb.de} schicken.

Happy \TeX ing\ldots

\enlargethispage{\baselineskip}


\minisec{Liste der Änderungen:}

\begin{labeling}[\hspace{\labelsep}--]{v0.52, 1999/07/04}
  \item[v0.51, 1999/03/02] New (LPPL) license
  \item[v0.52, 1999/07/04] Correction of wrong height calculation (if $\mathrm{depth} \ne 0$)
\end{labeling}

\end{document}
